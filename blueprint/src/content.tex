% In this file you should put the actual content of the blueprint.
% It will be used both by the web and the print version.
% It should *not* include the \begin{document}
%
% If you want to split the blueprint content into several files then
% the current file can be a simple sequence of \input. Otherwise It
% can start with a \section or \chapter for instance.

\section{Preliminaries}

\begin{lemma}
  \label{lem:odd_deg_red_imp_root}
  Let $F$ be a field, and suppose all nonlinear odd-degree polynomials over $F$ are reducible. Then every odd-degree polynomial over $F$ has a root.
\end{lemma}
\begin{proof}
  Fix an odd-degree polynomial $f$ over $F$, and proceed by induction on $\deg f$. If $\deg f=1$, then $f$ has a root in $F$. Otherwise, $f$ is reducible by assumption; write $f=gh$ where $\deg g,\deg h>0$. Since $\deg f=\deg g+\deg h$, wlog $\deg g$ is odd. Since $\deg g<\deg f$, the factor $g$ has a root in $F$ by induction, and therefore so does $f$.
\end{proof}

\begin{lemma}
  \label{lem:sylow_galois}
  Fix a prime $p$, and let $M/K$ be a separable Galois extension of degree $p^k\cdot a$, where $p\nmid a$. Then, for $0\leq j\leq k$, there are intermediate fields $K\leq L_0\leq\cdots\leq L_k\leq M$, with $[L_j:K]=p^j\cdot a$.
\end{lemma}
\begin{proof}
  Since $M/K$ is Galois, $|\Gal(M/K)|=p^k\cdot a$. A version of Sylow's first theorem says that each subgroup of order $p^j$ with $0\leq j<k$ is contained in a subgroup of order $p^{j+1}$. By induction, $\Gal(M/K)$ has a chain of subgroups $H_k\leq\cdots\leq H_0\leq \Gal(M/K)$ with $|H_j|=p^{k-j}$. By the Galois correspondence, $L_j=M^{H_j}$ are the desired subfields.
\end{proof}

\begin{lemma}
  \label{lem:deg_2_classify}
  Let $K$ be a field with $\ch K\neq2$. Then there is a bijection between the quadratic extensions of $K$ (up to $K$-isomorphism) and the set
  \[\left(\frac{K^*}{{(K^*)^2}}\right)\setminus\{1\cdot(K^*)^2\}\]
  given by the map $x(K^*)^2\to K(\sqrt{x})$.
\end{lemma}
\begin{proof}
  Consider the map $\Phi:x\to K(\sqrt{x})$ from $K^*$. We will show it fully respects the relation $x(K^*)^2=y(K^*)^2$; then $\Phi$ descends to a injective map out of the quotient $K^*/(K^*)^2$. In particular, if $x\notin(K^*)^2$, then $\Phi(x)=K(\sqrt{x})$ is not $K$-isomorphic to $K$, and is therefore a quadratic extension of $K$.

  Indeed, if $x(K^*)^2=y(K^*)^2$, then $x=a^2y$ for some $a\in K$, and so $K(\sqrt{x})\cong_K K(\sqrt{y})$ via $\sqrt{x}\to a\sqrt{y}$. Conversely, if $\varphi:K(\sqrt{x})\to K(\sqrt{y})$ is a $K$-isomorphism, then $\varphi(\sqrt{x})=a+b\sqrt{y}$ for some $a,b\in K$, and so $x=a^2+yb^2+2ab\sqrt{y}$. Comparing coefficients in the $K$-basis $\{1,\sqrt{y}\}$, either $a=0$ or $b=0$. Therefore, either $x=a^2y$ and so $x(K^*)^2=y(K^*)^2$, or $x=a^2$, in which case $K(\sqrt{y})\cong_K K(\sqrt{x})\cong_K K$; that is, $x,y\in(K^*)^2$.

  It remains to show all quadratic extensions of $K$ are $K$-isomorphic to some $L\in\im\Phi$. Fix a quadratic extension $L/K$, and let $\{1,\alpha\}$ be a $K$-basis for $L$; then $\alpha^2=a\alpha+b$ for some $a,b\in K$. Let $\beta=2\alpha-a$. Since $\ch K\neq 2$, $\alpha=(\beta+a)/2$, and so $L=K+\beta K=K(\beta)$. Now, we compute $\beta^2=a^2+4b$. Therefore $L\cong_K\Phi(a^2+4b)$ via $\beta\to\sqrt{a^2+4b}$.
\end{proof}

Note that we will only use that this map is well-defined and surjective, and not that it is injective (which was the most annoying part to show).

\section{Ordered Fields}

\begin{definition}
  An ordered field is a field with a total order on it such that $+$ and $*$ are monotone with respect to the order.
\end{definition}

`Monotonicity' means that, if $a\leq b$, then $a+c\leq b+c$ and, if in addition $c\geq0$, then $ac\leq bc$.

\begin{definition}
  A preordering on a field $F$ is a subsemiring of $F$ containing all squares but not containing $-1$.
\end{definition}

\begin{definition}
  An ordering on a field $F$ is a preordering $P\subset F$ such that $P\cup -P=F$.
\end{definition}

It is an easy check from the axioms of an ordered field that orderings deserve their name.

\begin{lemma}
  Linear orders on $F$ making $F$ into an ordered field correspond bijectively with orderings on $F$ via the map $(\leq)\to\{x\geq0\}$.
\end{lemma}

From now on, we will refer to the total order on an ordered field and its corresponding field ordering structure interchangeably.

Here is a basic result about ordered fields which uses the field ordering structure.

\begin{lemma}
  \label{lem:unique_ord_cond}
  Let $F$ be a real field. Suppose that, for each $a\in F$, either $a$ is a sum of squares or $-a$ is a sum of squares. Then there is a unique field ordering on $F$.
\end{lemma}
\begin{proof}
  Every ordering on $F$ contains the preordering given by sums of squares. By assumption, this preordering is in fact an ordering.
\end{proof}

\begin{corollary}
  \label{cor:unique_ord_cond_ordered}
  Let $F$ be an ordered field in which every non-negative element is a sum of squares. Then the ordering on $F$ is unique.
\end{corollary}

There is a convenient algebraic criterion for the existence of a field ordering.

\begin{definition}
  A field is (semi)real if $-1$ is not a sum of squares.
\end{definition}

\begin{lemma}
  \label{lem:preord_ext}
  Let $F$ be a real field. Then any preordering on $F$ extends to an ordering.
\end{lemma}

In particular, the preordering given by sums of squares extends to an ordering. 

\begin{theorem}
  \label{thm:ord_iff_real}
  A field can be ordered if and only if it is real.
\end{theorem}

Note that ordered fields have characteristic 0; in particular, their algebraic extensions are separable.

\begin{lemma}
  \label{lem:adj_sqrt_real}
  Let $F$ be an ordered field, and let $a\in F$ be non-negative. Then $F(\sqrt{a})$ is real.
\end{lemma}
\begin{proof}
  Suppose $-1$ is a sum of squares in $F(\sqrt{a})$; write $-1=\sum_i(a_i+b_i\sqrt{a})^2$ for some $a_i,b_i\in F$. We compute
  \[-1=\sum_i(a_i^2+ab_i^2)+2\sqrt{a}\sum_ia_ib_i.\]
  Comparing coefficients, $-1=\sum_i(a_i^2+ab_i^2)\geq0$.$_\#$
\end{proof}

\begin{lemma}
  \label{lem:odd_deg_real}
  Let $F$ be an ordered field, and let $K/F$ be an odd-degree extension. Then $K$ is real.
\end{lemma}
\begin{proof}
  Proceed by induction on $[K:F]$. By the primitive element theorem, $K=F(\alpha)$ for some $\alpha\in K$. Let $f$ be the minimal polynomial of $\alpha$ over $K$; then $\deg f=[K:F]$ is odd. Proceed by induction on $\deg f$; if $\deg f=1$, then $K\cong R$ is certainly real. Otherwise, suppose $-1+(f)=\sum_ig_i^2+(f)$ for some polynomials $g_i\in R[X]$ each of degree at most $\deg f-1$. Rearranging, $\sum_ig_i^2+1=hf$ for some $h\in R[X]$. Note that $h\neq0$ as $R$ is real. Let $a_i$ be the leading coefficient of $g_i$, and let $d=\max_i\deg g_i$; note that $0<d<\deg f$ by construction. Then
  \[\deg h+\deg f=\deg(\sum_i g_i^2+1)\leq 2d.\]
  If this inequality is strict, then, taking leading coefficients, $\sum_i a_i^2=0$ with the $a_i$ not all 0, again contradicting realness of $R$.$_\#$ Therefore $\deg h+\deg f=2d$. Then $\deg h$ is odd, so $h$ has an odd-degree irreducible factor $\tilde h$; we have
  \[\deg\tilde h\leq\deg h=2d-f<d.\]
  The field $F_{\tilde h}=F[X]/(\tilde h)$ has degree $\deg\tilde h$ over $F$; by induction, $F_{\tilde h}$ is real. However, we have $\sum_i g_i(X)^2+(\tilde h)=-1+(\tilde h)$.$_\#$ Therefore $K$ is real.
\end{proof}

Observe that, given an extension $L/K$ of ordered fields with total orders $(\leq_K),(\leq_L)$ and ordering structures $O_L,O_K$, we have $O_L\cap K=O_K$ if and only if $(\leq_L)|_K=(\leq_K)$. The meaning of a field ordering extending or restricting to another is therefore unambiguous.

\begin{lemma}
  \label{lem:ext_ord_to_real}
  Let $F$ be an ordered field, and let $K\supseteq F$ be real. Then the ordering on $F$ extends to an ordering on $K$.
\end{lemma}
\begin{proof}
  View the ordering on $F$ as a preordering on $K$. By Lemma \ref{lem:preord_ext}, this preordering extends to an ordering on $K$.
\end{proof}

\begin{definition}
  An extension $L/K$ of ordered fields is ordered if the ordering on $L$ restricts to the ordering on $K$.
\end{definition}

Lemmas \ref{lem:adj_sqrt_real} and \ref{lem:odd_deg_real} therefore allow us to construct ordered algebraic extensions. There is an easier way to construct ordered extensions if we don't care about them being algebraic.

\begin{lemma}
  \label{lem:order_fun_field}
  Let $F$ be an ordered field, and let $a\in F$. Then there is a unique ordering on the function field $F(X)$ extending the ordering on $F$ such that $X>a$ but $b>X$ for $b>a$, and a unique ordering such that $X<a$ but $b<X$ for $b<a$.
\end{lemma}
\begin{proof}
  Let $O$ be the set of polynomials over $F$ whose Taylor expansion at $a$ has positive lowest-order (nonzero) coefficient. It is straightforward to check that this is an ordering (note that, over a ring, we additionally need to check that, for $x\neq0$, $x$ and $-x$ are never both in $O$). This then extends to an ordering on $F(X)$ satisfying the required conditions.

  For the second construction, we instead take the polynomials with negative lowest-order coefficients in their Taylor expansions.
\end{proof}

Intuitively, $X$ is infinitesimally close to $a$. When $a=0$, we often write $R(\eps)$ for $R(X)$ with the first type of ordering.

\section{Real Closed Fields}

\begin{definition}
  A real closed field is an ordered field in which every non-negative element is a square and every odd-degree polynomial has a root.
\end{definition}

\begin{lemma}
  \label{lem:RCF_ord_unique}
  The ordering on a real closed field is unique.
\end{lemma}
\begin{proof}
  Follows directly from \ref{cor:unique_ord_cond_ordered}.
\end{proof}

In what follows, all algebraic extensions are given up to $R$-isomorphism, as is conventional. Observe that, since $-1$ is not a square in $R$, $R(i)/R$ is a quadratic extension. We show that this is the \textbf{only} nontrivial algebraic extension of $R$.

\begin{lemma}
  \label{lem:alg_ext_odd_deg}
  Nontrivial algebraic extensions of $R$ have even degree.
\end{lemma}
\begin{proof}
  Let $K/R$ be an odd-degree algebraic extension of $R$. By the primitive element theorem, $K=R(\alpha)$ for some $\alpha\in K$. Let $f$ be the minimal polynomial of $\alpha$ over $K$. Then $f$ is irreducible, but $\deg f=[K:R]$ is odd, so $f$ has a root in $R$. Therefore, $[K:R]=\deg f=1$; that is, $K\cong R$.
\end{proof}

\begin{lemma}
  \label{lem:ext_deg_2}
  The field $R(i)$ is the unique quadratic extension of $R$.
\end{lemma}
\begin{proof}
  Fix $x\in R^*$. Then either $x>0$ and $x=1\cdot(\sqrt{x})^2$, or $x<0$ and $x=-1\cdot(\sqrt{-x})^2$. Further, since $-1\notin(R^*)^2$, $-1\cdot(R^*)^2\neq 1\cdot(R^*)^2$. Therefore $R^*/{(R^*)^2}=\{1\cdot(R^*)^2,-1\cdot(R^*)^2\}$, and we are done by Lemma \ref{lem:deg_2_classify}.
\end{proof}

\begin{lemma}
  \label{lem:Ri_ext_deg_2}
  There is no quadratic extension of $R(i)$.
\end{lemma}
\begin{proof}
  By Lemma \ref{lem:deg_2_classify}, it suffices to show that every element of $R(i)$ is a square. Indeed, take $x=a+bi\in R(i)$ with $a,b\in R$. If $b=0$, then either $a\geq0$ and so $x$ is a square in $R$, or $a\leq0$ and so $a=(i\sqrt{-a})^2$ is a square in $R$. Now let $b\neq0$. Then we compute $x=(c+di)^2$, where
  \[c=\sqrt{\frac{a+\sqrt{a^2+b^2}}{2}}\text{ and }d=\frac{b}{2c}.\]
  To see that $c$ and $d$ are well-defined elements of $R$, observe that $a^2+b^2>a^2\geq0$ (as $b\neq0$), and so $a+\sqrt{a^2+b^2}>0$. Therefore the square roots above lie in $R$ and $c\neq0$.
\end{proof}

\begin{theorem}
  \label{thm:FTAlg}
  The only algebraic extensions of $R$ are $R$ itself and $R(i)$.
\end{theorem}
\begin{proof}
  By separability, every algebraic extension of $R$ is contained in a finite Galois extension. Since $R(i)/R$ has no intermediate fields, it suffices to show the result for finite Galois extensions.
  
  Let $K/R$ be a nontrivial Galois extension of degree $2^k\cdot a$, where $k\geq 0$ and $a\geq1$ is odd. By Lemma \ref{lem:sylow_galois} with $p=2$, there is an intermediate extension of degree $a$. By Lemma \ref{lem:alg_ext_odd_deg}, $a=1$ (and $k>0$). If $k>1$, then applying Lemma \ref{lem:sylow_galois} again yields intermediate extensions $K/L/M/R$ with $[L:M]=[M:R]=2$. By Lemma \ref{lem:ext_deg_2}, $M\cong R(i)$, contradicting Lemma \ref{lem:Ri_ext_deg_2}.$_\#$ Therefore $k=1$ and (by Lemma \ref{lem:ext_deg_2}) $K\cong R(i)$.
\end{proof}

\begin{corollary}
  $\bar{R}=R(i)$.
\end{corollary}

The converse to Theorem \ref{thm:FTAlg} is much easier.

\begin{lemma}
  \label{lem:FTAlg_converse}
  Suppose $R$ is an ordered field whose only nontrivial algebraic extension is $R(i)$. Then $R$ is real closed.
\end{lemma}
\begin{proof}
  Let $a\in R$ be non-negative, and consider the polynomial $f=X^2-a$. If $f$ is irreducible, then $R(\sqrt{a})\cong R(i)$. Suppose $i$ maps to $x+y\sqrt{a}$ for some $x,y\in R$; then $-1=x^2+ay^2+2xy\sqrt{a}$. Comparing coefficients, $-1=x^2+ay^2\geq0$.$_\#$ Therefore $f$ is reducible, and so $a$ is a square in $R$.

  Now, fix a nonlinear odd-degree polynomial $f\in R[X]$. Then $R[X]/(f)$ cannot be a field since $R$ has no nontrivial odd-degree extensions, and so $f$ must be reducible. We are done by Lemma \ref{lem:odd_deg_red_imp_root}.
\end{proof}

As before, let $R$ be a real closed field. Theorem \ref{thm:FTAlg} is a powerful tool for deriving more of its properties.

\begin{lemma}
  \label{lem:RCF_max}
  $R$ is maximal with respect to algebraic extensions by ordered fields.
\end{lemma}
\begin{proof}
  Since $-1$ is a square in $R(i)$, the field $R(i)$ is not formally real and therefore cannot be ordered. We are done by Theorem \ref{thm:FTAlg}.
\end{proof}

In particular, $R$ is maximal with respect to ordered algebraic extensions.

\begin{lemma}
  \label{lem:irreds_class}
  The monic irreducible polynomials over $R[X]$ have form $X-c$ for some $c\in R$ or $(X-a)^2+b^2$ for some $a,b\in R$ with $b\neq0$.
\end{lemma}
\begin{proof}
  Let $f\in R[X]$ be monic and irreducible. The field $R_f=R[X]/(f)$ is an algebraic extension of $R$, so it is classified by Theorem \ref{thm:FTAlg}. If $R_f\cong R$, then $\deg f=1$, so $f=X-c$ for some $c\in R$. If $R_f\cong R(i)$, let the isomorphism be $\varphi$, and suppose $\varphi(X+(f))=a+bi$ ($a,b\in R$). Note that $b\neq0$ since $\varphi^{-1}$ is constant on $R$. Rearranging, we see that $\varphi((X-a)^2+b^2+(f))=0$; that is, $(X-a)^2+b^2\in(f)$. Since this polynomial is monic and has the same degree as $f$, it must in fact be equal to $f$.

  Conversely, linear polynomials over a domain are irreducible by degree, and reducible quadratics have a root. A root of $f=(X-a)^2+b^2$ with $a,b\in R$ is an element $r\in R$ satisfying $(r-a)^2=-b^2$. Since squares are non-negative, if $b\neq0$ then $f$ must be irreducible.
\end{proof}

The next property is a little less obvious.

\begin{lemma}
  \label{lem:IVP_poly}
  $R$ satisfies the intermediate value property for polynomials.
\end{lemma}
\begin{proof}
  We will prove that, for all $f\in R[X]$ and all $a,b\in R$ with $a<b$, if $f(a)\cdot f(b)<0$, then there is some $c\in(a,b)$ such that $f(c)=0$.

  Fix $a,b\in R$ with $a<b$. First, suppose $f\in R[X]$ is linear. Then $f=m(X-c)$ for some $m,c\in R$ with $m\neq0$; then $f(c)=0$. If $m>0$, then $f(x)<0$ for $x<c$ and $f(x)>0$ for $x>c$, and vice versa if $m<0$. In either case, if $c\notin[a,b]$, then $f(a)\cdot f(b)>0$. Taking into account the cases $c=a$ and $c=b$, if $f(a)\cdot f(b)<0$ then $c\in(a,b)$.

  Now suppose $f(a)\cdot f(b)<0$, and proceed by induction on $\deg f$. If $\deg f=0$, write $f=x\in R$; then $f(x)\cdot f(x)=x^2\leq 0$, so, since squares are non-negative, $x=0$ and $f((a+b)/2)=0$. The above validates the property for $\deg f=1$. Now, take a monic irreducible factor $g$ of $f$; then $g$ is classified by Lemma \ref{lem:irreds_class}. If $g=(X-a)^2+b^2$ with $a,b\in R$ and $b\neq0$, then $g$ is everywhere positive. If $g=X-c$ with $c\in R$, then either $c\in(a,b)$ and $g(c)=0$, or $c\notin(a,b)$ and $g(a)$ and $g(b)$ have the same sign (they are nonzero since $f(a)$ and $f(b)$ are). In the second case, $f$ has a root in $(a,b)$; in the first and third cases, $f/g$ satsifies the induction hypothesis, so it has a root in $(a,b)$. In all cases, a factor of $f$ has a root in $(a,b)$, and therfore so does $f$.
\end{proof}

In fact, the converses to Lemmas \ref{lem:RCF_max} and \ref{lem:IVP_poly} also hold!

\begin{theorem}
  Let $R$ be an ordered field satisfying the intermediate value property for polynomials. Then $R$ is real closed.
\end{theorem}
\begin{proof}
  Let $a\in R$ be non-negative, and consider the polynomial $f=X^2-a$. Then $f(0)=-a\leq0$, but $f(a+1)=a^2+a+1>0$. By the intermediate value property, $f$ has a root in $R$, and so $a$ is a square in $R$.

  Let $f$ be an odd-degree polynomial over $R$. Write $f=a_nX^n+\cdots+a_0$. We will show $f$ has a root in $R$. Replacing $f$ by $-f$ if necessary, we may assume $a_n>0$. For $x>1$, we compute
  \[f(x)\geq x^{n-1}(a_nx-n\max_i|a_i|).\]
  Therefore, when $x>\max\{1,n\max_i|a_i|/a_n\}$, $f(x)>0$. A similar calculation shows that $f(x)<0$ for sufficiently large negative values of $x$. We are done by the intermediate value property.
\end{proof}

\begin{theorem}
  \label{thm:ord_max_imp_RCF}
  Let $R$ be an ordered field maximal with respect to algebraic extensions by ordered fields. Then $R$ is real closed.
\end{theorem}
\begin{proof}
  By Theorem \ref{thm:ord_iff_real}, $R$ is maximal with respect to algebraic extensions by \textit{real} fields.

  Let $a\in R$ be non-negative. By Lemma \ref{lem:adj_sqrt_real}, $R[\sqrt{a}]$ is real. By maximality, $R[\sqrt{a}]\cong R$, and so $a$ is a square in $R$.

  By Lemma \ref{lem:odd_deg_red_imp_root}, it suffices to show that irreducible odd-degree polynomials over $R$ are linear. Let $f\in R[X]$ be such a polynomial, and consider the odd-degree field extension $R_f=R[X]/(f)$. By Lemma \ref{lem:odd_deg_real}, $R_f$ is real; by maximality, $R_f\cong R$, and therefore $\deg f=[R_f:R]=1$.
\end{proof}

There is a subtlety here: we used the weaker form of maximality, disallowing algebraic extensions of $R$ by ordered fields with incompatible orders. In fact, such extensions cannot exist.

\begin{corollary}
  \label{cor:ord_max_imp_RCF_strong}
  Let $R$ be an ordered field maximal with respect to ordered algebraic extensions. Then $R$ is real closed.
\end{corollary}
\begin{proof}
  Let $a\in R$ be non-negative. By Lemmas \ref{lem:adj_sqrt_real} and \ref{lem:ext_ord_to_real}, there is an ordering on $R(\sqrt{a})$ making $R(\sqrt{a})/R$ into an ordered extension. By maximality, $R(\sqrt{a})\cong R$, and so $a$ is a square in $R$. By Corollary \ref{cor:unique_ord_cond_ordered}, $R$ is ordered uniquely.

  Let $K/R$ be an algebraic extension by an ordered field. The ordering on $K$ restricts to some field ordering on $R$; by uniqueness, this is \textit{the} ordering on $R$, and so the extension is ordered. By assumption, therefore, $R$ is maximal with respect to algebraic extensions by ordered fields. We are done by Theorem \ref{thm:ord_max_imp_RCF}.
\end{proof}

Theorem \ref{cor:ord_max_imp_RCF} gives us a way to ``construct'' real closed fields.

\begin{lemma}
  An algebraically closed field of characteristic 0 has an index-2 real closed subfield.
\end{lemma}
\begin{proof}
  Let $C$ be an algebraically closed field of characteristic 0. Observe that the prime subfield $\Q$ can be ordered. Further, given an ordered subfield $F$ with $\bar{F}\neq C$, we can use Lemma \ref{lem:order_fun_field} to adjoin an element transcendental over $F$, obtaining a strictly bigger ordered subfield.
  
  Apply Zorn's lemma to obtain a maximal ordered subfield $R\subseteq C$; then $\bar{R}=C$. By Theorem \ref{thm:ord_max_imp_RCF}, $R$ must be real closed. By Theorem \ref{thm:FTAlg}, $C\cong R(i)$, and so $[C:R]=2$.
\end{proof}

\section{Real Closures}

\begin{definition}
  Let $F$ be an ordered field. A real closure of $F$ is a real closed ordered algebraic extension of $F$.
\end{definition}

\begin{lemma}
  \label{lem:real_closure_exists}
  Let $F$ be an ordered field. Then $F$ has a real closure.
\end{lemma}
\begin{proof}
  Apply Zorn's lemma to ordered algebraic extensions of $F$. We are done by Corollary \ref{cor:ord_max_imp_RCF_strong}.
\end{proof}

Just like with the algebraic closure, it makes sense to talk of \textit{the} real closure of an ordered field. Proving this uniqueness result requires a method of root-counting in real fields known as Sturm's theorem.

\begin{theorem}[Corollary to Sturm's Theorem]
  \label{TODO: Sturm}
  Let $F$ be an ordered field, and let $f$ be a polynomial over $F$. Then $f$ has the same number of roots in any real closure of $F$.
\end{theorem}

\begin{lemma}
  \label{lem:closure_emb_ext_unordered}
  Let $F$ be an ordered field with a real closure $R$, and let $K/F$ be a finite ordered extension. Then there is an $F$-homomorphism $K\to R$.
\end{lemma}
\begin{proof}
  By the primitive element theorem, $K=F(\alpha)$ for some $\alpha\in K$. Let $f$ be the minimal polynomial of $\alpha$ over $F$. Since $F$ has a root in $K$, it has a root in a real closure of $K$ (one exists by \ref{lem:real_closure_exists}). By Theorem \ref{TODO: Sturm}, $f$ has a root $\beta$ in $R$. Therefore define $\varphi:K\to R$ with $\varphi(\alpha)=\beta$.
\end{proof}

\begin{lemma}
  \label{lem:closure_emb_ext}
  Let $F$ be an ordered field with a real closure $R$, and let $K/F$ be a finite ordered extension. Then there is a unique order-preserving $F$-homomorphism $K\to R$.
\end{lemma}
\begin{proof}
  Fix a real closure $R'$ of $K$.
  
  By the primitive element theorem, $K=F(\alpha)$ for some $\alpha\in K$. Let $f$ be the minimal polynomial of $\alpha$ over $F$, and let $\alpha_1<\cdots<\alpha_m$ be the roots of $f$ in a real closure $R'$ of $K$, with $\alpha=\alpha_k$. By Theorem \ref{TODO: Sturm}, $f$ also has $m$ roots in $R$; let them be $\beta_1<\cdots<\beta_m$. Since non-negative elements of $R'$ are squares, and $x_1,\dots,x_{m-1}\in R'$ such that $\alpha_{j+1}-\alpha_j=x_j^2$, and let $L=K(\alpha_1,\dots,\alpha_m,r,x_1,\dots,x_{m-1})\leq R'$. Now, suppose we have a $K$-homomorphism $\psi:L\to R$. Each $\psi(\alpha_j)$ is equal to a different $\beta_i$. Then $\psi(\alpha_{j+1})-\psi(\alpha_j)=\psi(x_j)^2\geq0$, so $\psi(\alpha_1)<\cdots<\psi(\alpha_m)$, and so $\psi(\alpha_j)=\beta_j$ for all $j$.

  By Lemma \ref{lem:closure_emb_ext_unordered}, there is in fact an $F$-homomorphism $\varphi:L\to R$. We will show that $\varphi$ is order-preserving. Indeed, fix a non-negative element $x\in L$. As before, find $r\in R'$ such that $x=r^2$, and let $M=L(r)\leq R'$. Apply Lemma $\ref{lem:closure_emb_ext_unordered}$ again to obtain an $L$-homomorphism $\psi:M\to R$; then $\varphi(x)=\psi(x)=\psi(r)^2\geq0$. Therefore $\varphi$ maps non-negative elements to non-negative elements, and so is order-preserving. Then $\varphi|_K$ is the map we want. Note that $\varphi(\alpha)=\beta_k$.

  To see uniqueness, let $\tilde\varphi:K\to R$ be an order-preserving $F$-homomorphism; by existence, $\tilde\varphi$ extends to a an order-preserving $K$-homomorphism $\tilde\psi:L\to R$. Then $\tilde\varphi(\alpha)=\tilde\psi(\alpha_k)=\beta_k=\varphi(\alpha)$, and so $\tilde\varphi=\varphi$.
\end{proof}

Taking $K=F$ above, we see that the order-embedding of a field into its real closure is unique.

\begin{theorem}
  \label{thm:real_closure_unique}
  Let $F$ be an ordered field. Then the real closure of $F$ is unique up to unique $F$-isomorphism.
\end{theorem}
\begin{proof}
  Let $R_1$ and $R_2$ be real closures of $F$. Applying Zorn's lemma to the set of ordered extensions intermediate between $R_1$ and $F$ having a unique order-preserving $F$-embedding into $R_2$, and using Lemma \ref{lem:closure_emb_ext}, we obtain an intermediate extension $R_1/K/F$ with no nontrivial finite ordered extensions and a unique order-preserving $F$-embedding $\varphi:K\to R_2$. Since the ordered algebraic extension induced by $\varphi$ contains an ordered finite extension, $\varphi$ must be an $F$-isomorphism. In particular, $K\subseteq R_1$ is real closed; by maximality (\ref{lem:RCF_max}), in fact $K=R_1$ and so $\varphi$ is an $F$-isomorphism.
\end{proof}

\begin{corollary}
  A real closed field has no nontrivial order-preserving automorphisms.
\end{corollary}

This uniqueness result is stronger than the one in the algebraically closed case: an algebraically closed field has many nontrivial automorphisms.

\section{The Artin-Schreier Theorem}

We didn't actually need to assume the ordering on $R$ in Lemma \ref{lem:FTAlg_converse}.

\begin{lemma}
  \label{lem:sumsq_is_sq}
  Let $F$ be a field, and suppose every element of $F(i)$ is a square. Then sums of squares in $F$ are squares in $F$.
\end{lemma}
\begin{proof}
  Given $a,b\in F$, find $c,d\in F$ such that $a+bi=(c+di)^2$ in $F(i)$. Then $a=c^2-d^2$ and $b=2cd$, so $a^2+b^2=(c^2+d^2)^2$ is a square in $F$. By induction, every sum of squares in $F$ is a square in $F$.
\end{proof}

\begin{lemma}
  \label{lem:FTAlg_converse_strong}
  Suppose $R$ is a field whose only nontrivial algebraic extension is $R(i)$. Then there is a unique field ordering on $R$, and moreover $R$ with this ordering is real closed.
\end{lemma}
\begin{proof}
  We are very nearly done by Theorem \ref{thm:ord_iff_real}, Lemma \ref{lem:FTAlg_converse}, and Corollary \ref{cor:unique_ord_cond_ordered}; it remains to show that $R$ is real.

  We have $\bar{R}\cong R(i)$. Since $R(i)\not\cong R$, $-1$ is not a square in $R$. Since, every element of $R(i)$ is a square, Lemma \ref{lem:sumsq_is_sq} says that $-1$ is not a sum of squares in $R$.
\end{proof}

In fact, we can go further. The following is a weak form of the Artin-Schreier theorem; removing the condition on characteristic is possible, but requires some more involved algebra.

\begin{theorem}
  \label{thm:Artin-Schreier_weak}
  Let $R$ be a field with $\ch K\neq2$, and suppose $[\bar{R}:R]=2$. Then there is a unique field ordering on $R$, and moreover $R$ with this ordering is real closed.
\end{theorem}
\begin{proof}
  By Lemma \ref{lem:FTAlg_converse_strong}, it suffices to show that $\bar{R}\cong R(i)$.

  By Lemma \ref{lem:deg_2_classify}, $\bar{R}\cong R(\sqrt{a})$ for some non-square $a\in R$. Since $R(\sqrt{a})$ is algebraically closed, $\sqrt{a}$ is a square in $R(\sqrt{a})$; find $x,y\in R$ such that $\sqrt{a}=(x+y\sqrt{a})^2$. Expanding and comparing coefficients, $x^2+y^2a=0$ and $2xy=1$. Rearranging, $a=-(4x^4)=-1\cdot(2x^2)^2$. By Lemma \ref{lem:deg_2_classify}, $R(i)\cong R(\sqrt{a})=\bar{R}$.
\end{proof}

We can weaken the hypotheses even further. Here is the full Artin-Schreier theorem.

\begin{theorem}[Artin-Schreier Theorem]
  \label{thm:Artin-Schreier}
  Let $R$ be a field, and suppose $\bar{R}$ is a \textit{finite} extension of $R$. Then there is a unique field ordering on $R$, and moreover $R$ with this ordering is real closed.
\end{theorem}
\begin{proof}
  TODO
\end{proof}

\begin{corollary}
  An algebraically closed field of nonzero characteristic has no finite index subfields.
\end{corollary}
\begin{proof}
  Ordered fields have characteristic 0.
\end{proof}

\begin{corollary}
  $\Q_\text{alg}$ has a finite index subfield unique up to $\Gal(\Q_{alg}/\Q)$.
\end{corollary}
\begin{proof}
  Any finite-index subfield must be real closed. Let $R$ be a real closed subfield of $\Q_\text{alg}$. Then the ordering on $R$ restricts to an ordering on $\Q$. Now, $R/\Q$ is algebraic, so $R$ is a real closure of $\Q$. Further, the field ordering on $\Q$ is unique. We are done by Lemma \ref{thm:real_closure_unique}.
\end{proof}